%% INTRODUCTION
%%

% Stellar characterization techniaues overview, traditionally frequentist

% who cares about characterization?

Accurate stellar characterization plays a vital role for many active research
fields within astronomy.
For example, stellar populations, galactic archaeology, the study of binary
stars, asteroseismology and exoplanet studies all rely on inferences of basic
stellar parameters to varying degrees.
% Empirically-derived and reliable estimates are of particular value for
% increasing our confidence in the end-product results built upon these inputs.

Basic stellar parameters such as effective temperature and surface gravity
can be inferred using one (or more) of several types of observations, e.g.
spectroscopy, photometry, interferometry, etc.
This inference can be performed using theoretical models or by building an
empirical calibration library.
For example, an observed stellar spectrum could be matched against a library of
theoretical spectra generated using stellar atmosphere models, or against a
library of observed spectra of ``standard stars'', serving as calibrators.
%Fundamental stellar parameters, such as mass and radius, which cannot be
%directly inferred from the available observations can often be inferred by
%invoking theoretical stellar models, e.g. fitting model isochrones to
%spectroscopic measurements.
Regardless of the approach, be it theoretical or empirical, the methods used
for the inference of stellar parameters are traditionally ``deterministic''.
A deterministic model can be loosely described as one where a particular
observational input always returns a single-valued output for a parameter of
interest.
In this framework, nature itself has no variance and our underlying model is
considered to be a perfect description of reality.

An alternative approach for inferring model parameters is a ``probabilistic''
one.
Unlike the deterministic case, a single observational input is intepreted to be
caused by a range of possible model parameters, described by a probability
distribution.
This statement is true even in the case of a perfect observation of infinite
signal-to-noise, since our underlying model itself comes with uncertainty.
In recent years there has been a shift towards such methods in several areas of
 astronomy and particularly within the exoplanet community.
For example, \citet{wolfgang:2015} considered the mass-radius relationship
of exoplanets to be probabilistic, since a planet of a particular radius could
have a range of masses thanks to variances in compositions, environment and
other circumstances.
These recent demonstrations in exoplanetary science prompt us to consider the
need for treating the parent stars in the same probabilistic framework, with
potential applications spanning many fields of astronomy.

The demand for probabilistic stellar parameters is not only motivated by the
fact that probability distributions are far more representative of our
`beliefs' about astrophysical parameters, it also has a practical purpose.
When using data published in the astronomical literature to, for example, infer
relationships between observed parameters, that inference can be performed as
the final stage in a hierarchical treatment.
This is useful if one wants to account for multi-dimensional uncertainties by
marginalizing over the `true' parameter values upon which the noisy
observations are dependent.
This point is discussed further in \textsection\ref{sec:HBM}.

%A number of recent exoplanet studies have used Hierarchical Bayesian Modeling
%(HBM) to model the dispersion, or intrinsic scatter in relations between
%astrophysical properties.
%For example, \citep{wolfgang:2015} use HBM to recalibrate the mass-radius
%relation for sub-Neptunes and infer the level of astrophysical dispersion
