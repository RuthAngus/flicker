%% INTRODUCTION
%%

% Stellar characterization techniaues overview, traditionally frequentist

% who cares about characterization?

Accurate and precise stellar characterization forms the bedrock of a number
fields within astronomy.
For example, stellar populations, galactic archaeology, binary stars,
asteroseismology and exoplanet studies all rely on inferences of fundamental
stellar parameters.
These stellar parameters are inferred from a variety of observable properties
measured using spectroscopy, photometry, interferometry, etc.
Many stellar parameters are inferred via theoretical stellar models, e.g.
fitting model isochrones to spectroscopic measurements, and
some are inferred using empirically calibrated relations.
Regardless of the approach, be it theoretical or empirical, the methods used
for the inference of stellar parameters are traditionally frequentist.
However, a shift towards probabilistic methods in other areas of astronomy,
and particularly within the exoplanet community, prompts the need to treat
stars in the same probabilistic framework.

The demand for probilistic stellar parameters not only motivated by the fact
that probability distributions are far more representative of our `beliefs'
about astrophysical parameters, but also... (ruth will continue this sentence)

%A number of recent exoplanet studies have used Hierarchical Bayesian Modeling
%(HBM) to model the dispersion, or intrinsic scatter in relations between
%astrophysical properties.
%For example, \citep{wolfgang:2015} use HBM to recalibrate the mass-radius
%relation for sub-Neptunes and infer the level of astrophysical dispersion
