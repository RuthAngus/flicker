%% DISCUSSION
%%

We have recalibrated the relation between short timescale brightness
fluctuations in the {\it Kepler} light curves of stars (flicker) with both
stellar density and surface gravity, whilst including parameters to describe
the intrinsic scatter in these relationships.
We included both an additive variance term {\it and} a term that allows for
some abscissa-dependent variance, presented in table \ref{tab:table}.
The additive terms, $\sigma_\rho$ and $ \sigma_g$ are both non-zero,
suggesting that there {\it is} an additional source of scatter in the
relations, not accounted for by the observational uncertainties alone.
This is either caused by intrinsic scatter in the physical relationship
between flicker and density and \logg, produced by some physical process that
is not accounted for in the model, or by an underestimation of the
observational uncertainties.
Interestingly, the abscissa-dependent variances, $\gamma$ and $\zeta$ are both
less than zero.
This indicates that the intrinsic scatter either decreases with increasing
flicker and decreasing stellar density and surface gravity, or that the
uncertainty underestimation is more severe for the giant stars.
Regardless of the cause of the additional scatter needed to describe the
relations between these parameters, the conclusions of this study are the same.
Firstly, naively modelling the relationship between flicker and surface
gravity or stellar density, or any other stochastic relationship
deterministically, will lead to an underprediction of the uncertainties in
dependent variable.
Secondly, we find no evidence for a increased scatter at higher surface
gravities and densities, i.e. the relations between flicker, surface gravity
and stellar density are equally valid for dwarfs and giants alike.

This is a simple `fitting a line to data' exercise, however it continues the
discussion of probabilistic modelling that is an active topic within the
fields of exoplanet and stellar astronomy.
We use Hierarchical Bayesian Modelling (HBM) to constrain the intrinsic
scatter in the relationship between flicker, surface gravity and density.
We also include the effects of the non-negligable two-dimensional observational
uncertainties by marginalizing over latent variables using importance sampling.
% Although none of these methods, nor the data used here are new taken alone,
% we use the example of flicker to demonstrate the ideas and the importance of
% probabilistic, hierarchical inference.
Relationships between astronomical parameters are almost always
non-deterministic; an element of stochasticity effects the physical parameters
of stars so one can never perfectly predict $y$ given an observation of $x$.
We advocate a probabilistic approach in both the `fitting the model to data'
step, {\it and} when using an empirically calibrated model to predict
parameter values.
The fitting stage benefits because if the relationships between parameters are
falsely assumed to be deterministic, they will be skewed by data points with
unrepresentative uncertainties.
The prediction stage benefits from the stochastic treatment both because a
probability distribution is in many ways more representative of an observation
than a point estimate, and because posterior PDF samples can be used in
subsequent studies (provided the prior used during the fitting process is
described).

We provide posterior PDF samples and the code used in this project at
\url{https://github.com/RuthAngus/flicker}.
Whenever a prediction for the surface gravity or density of a star is required,
for a given estimate of flicker, we recommend using these posterior samples
within the calculation of $\rho_{*}$ or $\log(g)$ and its (Monte Carlo)
uncertainty.
These posterior samples will naturally fold-in the covariances between
parameters.
Simple analytical uncertainty propagation is only valid when uncertainties are
Gaussian and uncorrelated which is rarely true and certainly not the case when
the model is a straight line (the slope and intercept are alway correlated).
A flicker value with uncertainties (or even better: posterior PDF
samples), input into our model will result in a probability distribution over
stellar densities or surface gravities which reflects both the uncertainties
on the flicker measurement, the uncertainties on the model parameters {\it and}
the intrinsic scatter in the flicker-$\rho_{*}$-$\log(g)$ relations.
