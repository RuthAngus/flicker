%% HBM
%%

\subsection{Calibration Data}

For our calibration of \rhostar-$F_8$, we used a sample of Kepler stars with
both asteroseismic and flicker measurements available. \citet{chaplin:2014}
report asteroseismic \rhostar\ estimates (and the associated uncertainties) for 
518 Kepler stars. The authors report three different sets of results, depending 
on the choice of \Teff\ and \FeH, and in this work we elected to use values 
reported in their Table 6 over Table 5, and Table 5 over Table 4. We 
additionally used the 71 additional planet hosting stars with asteroseismology 
reported in \citet{huber:2013} but not reported in \citet{chaplin:2014}. Values 
for flicker and range were taken from \citet{kipping:2014}, based upon the 
methods described in \citet{bastien:2013}. Following \citet{kipping:2014} and 
for reasons described there-in, we only include targets in our calibration for 
which:

\begin{itemize}
\item[{\tiny$\blacksquare$}] Range (defined in \citealt{bastien:2013}) 
$<1$\,ppt
\item[{\tiny$\blacksquare$}] $4500<T_{\mathrm{eff}}<6500$\,K
\item[{\tiny$\blacksquare$}] $K_P<14$
\item[{\tiny$\blacksquare$}] $1.2 < \log_{10}$($F_8$\,[ppm])$< 2.2$
\end{itemize}

We use the same sample for our calibration of \logg, except that we exclude the
\citet{huber:2013} data, since these authors do not provide estimates of
\logg\footnote{Whilst we could compute \logg\ ourselves from the reported
masses and radii, this could only be done under the incorrect assumption of
zero covariance between $M_{\star}$ and $R_{\star}$.}

\subsection{Hierarchical Bayesian Model}

TBD

%%% rhostar plot
\begin{figure}
\begin{center}
\includegraphics[width=8.4cm,angle=0,clip=true]{../figs/rho_vs_flicker.pdf}
\caption{
TBD
} 
\label{fig:rhostar}
\end{center}
\end{figure}

%%% logg plot
\begin{figure}
\begin{center}
\includegraphics[width=8.4cm,angle=0,clip=true]{../figs/logg_vs_flicker.pdf}
\caption{
TBD
} 
\label{fig:logg}
\end{center}
\end{figure}
