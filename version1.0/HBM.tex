%% HBM
%%

\subsection{Calibration Data}

For our calibration of \rhostar-$F_8$, we used a sample of Kepler stars with
both asteroseismic and flicker measurements available. \citet{chaplin:2014}
report asteroseismic \rhostar\ estimates (and the associated uncertainties) for
518 Kepler stars. The authors report three different sets of results, depending
on the choice of \Teff\ and \FeH, and in this work we elected to use values
reported in their Table 6 over Table 5, and Table 5 over Table 4. We
additionally used the 71 additional planet hosting stars with asteroseismology
reported in \citet{huber:2013} but not reported in \citet{chaplin:2014}. Values
for flicker and range were taken from \citet{kipping:2014}, based upon the
methods described in \citet{bastien:2013}. Following \citet{kipping:2014} and
for reasons described there-in, we only include targets in our calibration for
which:

\begin{itemize}
\item[{\tiny$\blacksquare$}] Range (defined in \citealt{bastien:2013})
$<1$\,ppt
\item[{\tiny$\blacksquare$}] $4500<T_{\mathrm{eff}}<6500$\,K
\item[{\tiny$\blacksquare$}] $K_P<14$
\item[{\tiny$\blacksquare$}] $1.2 < \log_{10}$($F_8$\,[ppm])$< 2.2$
\end{itemize}

We use the same sample for our calibration of \logg, except that we exclude the
\citet{huber:2013} data, since these authors do not provide estimates of
\logg\footnote{Whilst we could compute \logg\ ourselves from the reported
masses and radii, this could only be done under the incorrect assumption of
zero covariance between $M_{\star}$ and $R_{\star}$.}

\subsection{Hierarchical Bayesian Model}

We model the stochastic relationship between flicker, log($g$) and stellar
density, accounting for the fact that there exists some intrinsic scatter in
the dependent variable, and including the heteroskedastic uncertainties on both
the dependent and independent variables.
There are two excellent reasons for modelling the relation stochastically;
firstly, if the intrinsic scatter is ignored and the relation between
variables is assumed to be deterministic, those data points with smaller
measurement uncertainties may have an unrepresentative greater weighting
during the fitting process.
Secondly, we are interested in producing probability distributions over stellar
densities and surface gravities, as opposed to point estimates, and propagating
these probability distributions through to subsequent analyses.
Several recent studies have required posterior Probability Distribution
Function (PDF) samples, in order to conduct their (hierarchical)
analyses \citep[e.g.][]{Rogers2014, Foreman-Mackey2014, Angus2015}.
Including observational uncertainties on the independent variable within our
analysis is also important, because ordinary least squares regression methods
performed on data with two-dimensional uncertainties will result in a slope
that is biased towards zero
(e.g., Fuller 1987; Akritas \& Bershady 1996; Fox 1997).
or a demonstration of the effects of neglecting intrinsic scatter and
two-dimensional uncertainties, see Kelly (2007).

The two models describing the relationship between flicker, logg and stellar
density can be written as
\begin{equation}
	\log(g) \sim \mathcal{N}(\mu = \alpha + \beta F_8, \sigma^2 = \
	\sigma^2 + \gamma F_8)
\end{equation}
\label{eq:logg}
and
\begin{equation}
	\rho_* \sim \mathcal{N}(\mu = \delta + \epsilon F_8, \sigma^2 = \
	\sigma^2 + \zeta F_8),
\end{equation}
\label{eq:rho}
where $F_8$ is flicker and $\rho_*$ is stellar density.
The free parameters of the two models are $\alpha$, $\beta$, $\sigma_G$,
$\gamma$, $\delta$, $\epsilon$ and $\zeta$.
The two relationships can be described as Gaussian distributions with means
given by the equation of a straight line, and standard deviations given by
observational uncertainties, plus some dispersion which is a function of the
dependent variable.

%%% rhostar plot
\begin{figure}
\begin{center}
\includegraphics[width=8.4cm,angle=0,clip=true]{../figs/rho_vs_flicker.pdf}
\caption{
TBD
}
\label{fig:rhostar}
\end{center}
\end{figure}

%%% logg plot
\begin{figure}
\begin{center}
\includegraphics[width=8.4cm,angle=0,clip=true]{../figs/logg_vs_flicker.pdf}
\caption{
TBD
}
\label{fig:logg}
\end{center}
\end{figure}
