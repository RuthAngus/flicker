%\documentclass[12pt,preprint]{aastex}
\documentclass[apjl]{emulateapj}

\pdfoutput=1

\usepackage{comment}
\usepackage{ifthen}
\usepackage{amsmath}
\usepackage{amsfonts}
\usepackage{amssymb}
\usepackage{multirow}
\usepackage{wasysym}
\usepackage{subfigure}
\usepackage{epsfig}
\usepackage[%
	colorlinks=true,	% false: boxed links; true: colored links
	urlcolor=blue,		% color of external links
	pdfpagelabels=true,
	hypertexnames=true,
	plainpages=false,
	naturalnames=true,
%	bookmarks=true,         % show bookmarks bar?
%	unicode=false,          % non-Latin characters in Acrobat’s bookmarks
%	pdftoolbar=true,        % show Acrobat’s toolbar?
%	pdfmenubar=true,        % show Acrobat’s menu?
%	pdffitwindow=false,     % window fit to page when opened
%	pdfstartview={FitH},    % fits the width of the page to the window
	pdftitle={Priors for Orbital Eccentricity},    % title
	pdfauthor={David M. Kipping},     % author
%	pdfsubject={Subject},   % subject of the document
%	pdfcreator={Creator},   % creator of the document
%	pdfproducer={Producer}, % producer of the document
%	pdfkeywords={keyword1} {key2} {key3}, % list of keywords
%	pdfnewwindow=true,      % links in new window
	linkcolor=red,          % color of internal links (change box color with linkbordercolor)
	citecolor=green,        % color of links to bibliography
%	filecolor=magenta,      % color of file links
        ]{hyperref}
%

% #############################################################################
% Specific latex definitions for this paper


%% -----------------------------------------------
%% Editing
\newcommand{\ms}{$\mathrm{m}\,\mathrm{s}^{-1}$}
\newcommand{\kms}{$\mathrm{km}\,\mathrm{s}^{-1}$}
\newcommand{\luna}{{\tt LUNA}}
\newcommand{\multi}{{\sc MultiNest}}
\newcommand{\cofiam}{{\tt CoFiAM}}
\newcommand{\blender}{{\tt BLENDER}}
\newcommand{\eccsamples}{{\tt ECCSAMPLES}}
\newcommand{\kepler}{\emph{Kepler}}
\newcommand{\Kepler}{\emph{KEPLER}}
\newcommand{\obs}{\mathrm{obs}}
\newcommand{\tru}{\mathrm{true}}
\newcommand{\kic}{KIC-8800954}
\newcommand{\koi}{KOI-1274}
\newcommand{\kep}{Kepler-421}
\newcommand{\koix}{KOI-1274.01}
\newcommand{\kepx}{Kepler-421b}




%% #############################################################################

%% VARIABLE DEFINITIONS
%%
\newboolean{emulateapj}
\setboolean{emulateapj}{true}

\newboolean{astroph}
\setboolean{astroph}{true}

\newboolean{png}
\setboolean{png}{false}

\newboolean{color}
\setboolean{color}{true}

%% ############################################################################

\shortauthors{Angus \& Kipping}
\shorttitle{Probabilistic Inference of Stellar Parameters}
\ifthenelse{\boolean{emulateapj}}{
    \newcommand{\titledag}{$\dagger$}
}{
    \newcommand{\titledag}{\dagger}
}
\def\mathbi#1{\textbf{\em #1}}

\begin{document}

%% Titlepage
\title {Probabilistic Inference of Basic Stellar Parameters:\\
Application to Flickering Stars %to Surface Gravity and Mean Density
\altaffilmark{\titledag}}

%% Authors
\author{
	{\bf	R.~Angus\altaffilmark{1,2},
		D.~M.~Kipping\altaffilmark{2,3},
	}
}

\altaffiltext{1}{Dept. of Physics, University of Oxford, UK; email:
		 ruth.angus@astro.ox.ac.uk}

\altaffiltext{2}{Harvard-Smithsonian Center for Astrophysics,
		Cambridge, MA 02138, USA; email: dkipping@cfa.harvard.edu}

\altaffiltext{3}{Department of Astronomy, Columbia University, 550 W 120th St.,
                 New York, NY 10027, US}

\altaffiltext{$\dagger$}{
%%
Based on archival data of the \emph{Kepler} telescope.
}


%% EOF authors

% #####################################################################
%% abstract
\begin{abstract}

% aims
The relations between observable stellar parameters are usually assumed to be
deterministic, that is, given an infinitely precise measurement of independant
variable, `$x$', and some model, the value of dependant variable, `y' can be
known exactly.
In practise this assumption is rarely valid and intrinsic stochasticity means
that two stars with exactly the same `$x$', will have slightly different
`$y$`s.
% methods
The relation between short-timescale brightness fluctuations (flicker) of stars
and both stellar density and surface gravity \citep{bastien:2013} are two such
stochastic relations.
We recalibrate these relations in a probabilistic framework, using
Hierarchical Bayesian Modelling (HBM) and constrain the instrinsic scatter in
the relations.
Our treatment also fully accounts for the observational uncertainties on all
variables.
% results
These new relations provide probability distributions over values of $\log(g)$
or $\rho_*$ for given measurements of flicker.
Such probability distributions are both more representative of what we know
about stars and will be useful for subsequent studies using probabilistic
inference.

\end{abstract}

% #####################################################################
%% keywords
\keywords{
%%
	stars: fundamental parameters --- techniques: photometric ---
        methods: statistical
%%
}

%% EOF keywords
%% EOF titlepage

% #####################################################################
%% Introduction
\section{INTRODUCTION}
\label{sec:intro}

%% INTRODUCTION
%%

Accurate stellar characterization plays a vital role for many active research
fields within astronomy. For example, stellar populations, galactic
archaeology, the study of binary stars, asteroseismology and exoplanet studies
all rely on inferences of basic stellar parameters to varying degrees.
Empirically-derived and reliable estimates are of particular value, increasing
our confidence in the end-product results built upon these inputs.

Basic stellar parameters, such as effective temperature and surface gravity,
can be inferred using one (or more) of several types of observations, such as
spectroscopy, photometry, interferometry, etc. This inference can be performed
by invoking theoretical models or by building an empirical calibration
library.
For example, an observed stellar spectrum could be matched against a library of
theoretical spectra generated using stellar atmosphere models, or, against a
library of observed spectra of ``standard stars'', serving as calibrators.
%Fundamental stellar parameters, such as mass and radius, which cannot be
%directly inferred from the available observations can often be inferred by
%invoking theoretical stellar models, e.g. fitting model isochrones to
%spectroscopic measurements.
Regardless of the approach, be it theoretical or empirical, the methods used
for the inference of stellar parameters are traditionally ``deterministic''.
In this context, a deterministic model can be loosely described as one where
a particular observational input always returns a single-valued output for a
parameter of interest, i.e. nature itself has no variance and the underlying
model is considered to be a perfect description of reality.

An alternative approach for inferring model parameters is to allow
relationships between observables to be stochastic.
Unlike the deterministic case, a single observational input is intepreted to
be caused by a range of possible model parameters, described by a probability
distribution.
This statement is true even in the case of a perfect observation of infinite
signal-to-noise, since the underlying model itself comes with uncertainty.
In recent years, there has been a shift towards such methods in several areas
of astronomy, particularly within the exoplanet community.
For example, \citet{wolfgang:2015} considered that the mass-radius
relationship of exoplanets is stochastic, since a particular sized planet
could be have a range of planet masses due to unmodeled variances in
compositions, environment and other complications.
These recent demonstrations in exoplanetary science have prompted us to
consider the need for treating the parent stars in the same probabilistic
framework, with potential applications spanning many fields of astronomy.

The demand for probilistic stellar parameters is not only motivated by the fact
that probability distributions are far more representative of our `beliefs'
about astrophysical parameters, it also has a practical purpose.
When using data published in the astronomical literature to, for example, infer
relationships between observed parameters, that inference can be performed as
the final stage in a hierarchical treatment
\citep[see, e.g.][]{foreman-mackey:2014}.
This is useful if, for example, one wants to account for multi-dimensional
uncertainties by marginalizing over the `true' parameter values upon which the
noisy observations are conditioned.
This point is discussed further in \textsection\ref{sec:HBM}.

%A number of recent exoplanet studies have used Hierarchical Bayesian Modeling
%(HBM) to model the dispersion, or intrinsic scatter in relations between
%astrophysical properties.
%For example, \citep{wolfgang:2015} use HBM to recalibrate the mass-radius
%relation for sub-Neptunes and infer the level of astrophysical dispersion

% RA text...

One of the more recent tools developed to characterize stars is known as
``flicker'' \citep{bastien:2013}.
Flicker is a proxy for the scatter on an 8-hour timescale (denoted as $F_8$)
in a broad visible bandpass time series photometric light curve, such as that
from \textit{Kepler} or the upcoming TESS mission. A more detailed account of
the proceedure to calculate flicker is described in \citet{bastien:2013}. As
shown in \citet{bastien:2013}, flicker displays a remarkable correlation to
the asteroseismically determined parent star surface gravities (\logg).
Turning this around, the observation implies that flicker can be used to
blindly infer surface gravities at the level of $\sim0.1$\,dex, an attractive
proposition given the wealth of photometric light curves available through the
array of exoplanet transit missions flying and scheduled to launch.

\citet{cranmer:2014} demonstrated that models of stellar surface granulation
indeed reproduce a flicker effect in close agreement with that observed by
\citet{bastien:2013}, providing a physically-plausible explanation.
Since surface gravity is highly correlated with mean stellar density (\rhostar)
on evolutionary tracks, \citet{kipping:2014} showed that flicker can be also
be used to infer \rhostar, which is more useful for exoplanet transit analysis
\citep{seager:2003}.

Whether one calibrates flicker to \logg\ or \rhostar, there are several aspects
of the problem which are attractive for our purposes of a simple demonstration
of probabilistic inference of stellar parameters.
Firstly, in log-log space the relationship is very simple, appearing to be
linear \citep{kipping:2014}.
Secondly, there is a sufficiently large number of points in the sample (439
stars) to constrain a population-based model.
Thirdly, there is significant excess scatter around the best-fitting relation
implying that a deterministic model is inadequate.
This is not surprising given that granulation is a complex and messy process
for which one should not expect any parametric model to provide a perfect
description.
Fourthly, the physical processes that produce surface granulation, of which
flicker is an observational tracer, may be more or less noisy for different
types of stars.
We will test whether flicker has greater predictive power in certain regions
of parameter space; i.e. is flicker significantly more informative for
subgiants than for dwarfs?
For these reasons, we identify the calibration of
flicker to \logg\ and \rhostar\ as a well-posed problem to first demonstrate
probabilistic inference in the arena of stellar characterization.


One of the more recent tools developed to characterize stars is known as
``flicker'' \citep{bastien:2013}. Flicker is a proxy for the scatter on an
8-hour timescale (dubbed $F_8$) in a broad visible bandpass time series
photometric light curve, such as that from \textit{Kepler} or the upcoming TESS
mission. A more detailed account of the proceedure to calculate flicker is
described in \citet{bastien:2013}. As shown in \citet{bastien:2013}, flicker
displays a remarkable correlation to the asteroseismically determined parent
star surface gravities (\logg). Turning this round, the observation implies
that flicker can be used to blindly infer surface gravities at the level of
$\sim0.1$\,dex, an attractive proposition given the wealth of photometric light
curves available through the array of exoplanet transit missions flying and
scheduled to launch.

\citet{cranmer:2014} demonstrated that stellar surface granulation is expected
to reproduce a flicker effect in close agreement with that observed by
\citet{bastien:2013}, providing a physically-plausible explanation. Since
surface gravity is highly correlated to mean stellar density (\rhostar) on
evolutionary tracks, \citet{kipping:2014} showed that flicker can be also
be used to infer \rhostar, which is more useful for exoplanet transit
analysis.

Whether one calibrates flicker to \logg\ or \rhostar, there are several aspects
of the problem which are attractive for our purposes of a simple demonstration
of probabilistic inference of stellar parameters. Firstly, in log-log space the
relationship is very simple, appearing to be linear. Secondly, there is a
sufficiently number of points in the sample (439 stars) to constrain a
population-based model. Thirdly, there is significant excess scatter around the
best-fitting relation implying that a deterministic model is inadequate. This is
not surprising given that granulation is a complex and messy process for which
one should not expect any parametric model to provide a perfect description. For
these reasons, we identify the calibration of flicker to \logg\ and \rhostar\
as a well-posed problem to first demonstrate probabilistic inference in the
arena of stellar characterization.

\section{PROBABILISTIC CALIBRATION}
\label{sec:HBM}

%% HBM
%%

TBD

%%% rhostar plot
\begin{figure}
\begin{center}
\includegraphics[width=8.4cm,angle=0,clip=true]{../figs/rho_vs_flicker.pdf}
\caption{
TBD
} 
\label{fig:rhostar}
\end{center}
\end{figure}


\section{APPLICATION}
\label{sec:appication}

%% APPLICATION
%%

We have presented a HBM to infer the probability distribution of a star's
surface gravity and bulk density using flicker. These models are calibrated by a 
comparison of flicker to asteroseismic values. In this section, we provide a
comparison of the predicted $\rho_{\star}$ values from our model versus an
independent metric, in order to validate our predictictions.

A suitable comparison comes from stars with transiting planets, where the shape 
of the transit light curve may also be used to derive $\rho_{\star}$. Several
effects can change 


\section{DISCUSSION}
\label{sec:discussion}

%% DISCUSSION
%%

TBD


% #####################################################################
%% Acknowledgements
\acknowledgements
\section*{Acknowledgements}

RA thanks...
DMK thanks...

%% EOF Acknowledgements

% #####################################################################
%% Bibliography
\begin{thebibliography}{99}
\bibitem[\protect\citeauthoryear{Angus et al.}{2015}]{angus:2015}
Angusr, R., Aigrain, S., Foreman-Mackey, D. \& McQuillan, A., 2015, MNRAS,
450, 1787
\bibitem[\protect\citeauthoryear{Bastien et al.}{2013}]{bastien:2013} Bastien,
F. A., Stassun, K. G., Basri, G. \& Pepper, J., 2013, Nature, 500, 427
\bibitem[\protect\citeauthoryear{Chaplin et al.}{2014}]{chaplin:2014}
Chaplin, W.~J., Basu, S., Huber, D. et al., 2014, ApJS, 210, 1
\bibitem[\protect\citeauthoryear{Cranmer et al.}{2014}]{cranmer:2014}
Cranmer, S. R., Bastien, F. A., Stassun, K. G. \& Saar, S. H., 2014, ApJ, 781,
124
\bibitem[\protect\citeauthoryear{Foreman-Mackey et al.}{2014}]
{foreman-mackey:2014} Foreman-Mackey, D., Hogg, D.~W. \& Morton, T.~D., 2014,
ApJ, 795, 64
\bibitem[\protect\citeauthoryear{Hogg et al.}{2010}]{hogg:2010}
Hogg, D.~W., Myers, A.~D. \& Bovy, J., 2010, ApJ, 725,
2166
\bibitem[\protect\citeauthoryear{Huber et al.}{2013}]{huber:2013}
Huber, D., Chaplin, W.~J., Christensen-Dalsgaard, J. et al., 2013, ApJ, 767,
127
\bibitem[\protect\citeauthoryear{Kipping et al.}{2014}]{kipping:2014}
Kipping, D.~M., Bastien, F.~A., Stassun, K.~G., Chaplin, W.~J., Huber, D. \&
Buchhave, L.~A., 2014, ApJ, 785, 32
\bibitem[\protect\citeauthoryear{Rogers}{2015}]{rogers:2015}
Rogers, L.~A., 2015, ApJ, 801, 41
\bibitem[\protect\citeauthoryear{Wolfgang et al.}{2015}]{wolfgang:2015}
Wolfgang, A., Rogers, L.~A., Ford, E.~B., 2015, ApJ, submitted
(astro-ph:1504.07557)
\end{thebibliography}

\end{document}

%% EOF document
